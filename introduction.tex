\todo{CS4002}
\todo[color=blue!40]{CS4003}
\todo[color=green!40]{CS4004}
La Introducción a un trabajo de investigación es donde configura su tema y enfoque para el lector. Tiene varios objetivos clave como presentar el tema y hace que el lector se interese, proporcionar antecedentes o resumir la investigación existente (revisión de la literatura), posicionamiento del enfoque del autor (crítica), detallar el problema de investigación específico. Por último, dar una descripción general de la estructura del documento.
\end{itemize}

Construye la sección Introducción a  partir de los siguientes movimientos:
\begin{enumerate}
  \item \textbf{Establecer el territorio de investigación} Basado en dos pasos: evidenciar la significancia del área y revisar la literatura.
  \item \textbf{Establecer el nicho de investigación} Justificar el tópico de investigación.
  \item \textbf{Colocar tu investigación dentro del nicho de investigación} Basado en los siguientes pasos: objetivos y alcance de tu investigación, definición de términos clave (opcional) y proporcionar el esquema del documento.
\end{enumerate}

\begin{tcolorbox}[colback=blue!5!white,colframe=blue!75!black,title=Ejemplo]
  Cuando se escribe la introducción, generalmente se sigue la estrategia del embudo. Se empiza de manera genera y se termina describiendo - de manera suscinta - la solución planteada. Imaginemos que deseamos analizar la propagación del SARS-CoV-2 utilizando algoritmos de propagación de rumores en redes sociales (grafos). Para este ejemplo, podemos empezar describiendo qué es la COVID-19 -de manera general -. Luego, se pueden discutir sobre la necesidad de utilizar una estructura de datos para modelar el SARS-CoV-2 . Posteriormente, se puede hablar sobre los grafos y cómo ellos nos pueden ayudar a comprender la propagación de esta enfermedad. Finalmente, podemos hablar de cómo utilizaremos un grafo dirigido para modelar la propagación y finalmente decirmos que utilizaremos algorirmos de propagación de rumores en grafos y así  poder comprender como la COVID-19 se propaga - específico -.
\end{tcolorbox}
