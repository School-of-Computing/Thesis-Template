En esta parte se desarrollan los conceptos necesarios para poder entender el problema. Por ejemplo, si el problema es modelar la propagación del COVID-19 mediante el uso de grafos, el marco teórico describirá los conceptos relacionados con COVID-19, de la necesidad de modelar este tipo de fenómenos. También se pueden hablar acerca de la propagación de enfermendades, etc. El marco teórico nos ayudará a seleccionar las palabras clave que serán usadas al momento de construir el estado del arte.

Por otro lado, también existe el marco conceptual, el cual nos ayudará a definir los conceptos acerca de la teoría relacionada con la solución que planteamos para solucionar el problema. En nuestro ejemplo anterior, el marco teórico cubrirá conceptos como, qué es un grafo, qué tipos de grafo existen, qué es densidad de un grafo o los concetos de propagación de rumores en grafos dirigidos. Generalmente, estos conceptos son descubiertos - o aprendidos - al momento de revisar la literatura asociada a nuestro problema.