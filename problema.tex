En esta sección se describe el problema a ser analizado. El problema debe ser claramente descrito, utilizando fuentes bibliográficas que nos ayuden a mostrar la importancia de su estudio. La descripción del problema debe ser impactante y si es posible debe esta acompañado de datos cuantitativos. En nuestor ejemplo, podemos hablar del impacto - en términos de contagios y decesos - que la COVID-19 en la población mundial. Luego podemos hablar de porqué es importante modelar la propagación de la enfermedad mediante estructuras de datos. En esta parte se sugiere utilizar alguna herramienta que permita esquematizar el problema a estudiar, por ejemplo, construir un árbol del problema y diseñar un diagrama de Ishikawa o cola de pez.

Esta sección puede complementarse con una justificación de ser necesario.