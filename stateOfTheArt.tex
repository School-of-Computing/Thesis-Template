El estado del arte o revisión bibliográfica permite pocisionar nuestro proyecto dentro de los trabajos ya existentes en la lieratura científica. En la revisión bibliográfica se deben revisar solamente documentos pertenecientes a la literatura primaria y secundaria, evitando la literatura terciaria y gris y la literatura no científica. Es importante definir el formato de la citaciones (e.g., APA) y las forma correcta de hacerlo. 

 Existen varias técnicas para construir esta parte de un proyecto de investigación. Podemos utilizar una técnica poco formal, como la Revisión Empírica o podemos utilizar una técnica más estructurada como la Revisión Sistemática~\cite{moreno2018revisiones}

La revisión de la literatura debe concluir con un resumen y una pequeña discusión.